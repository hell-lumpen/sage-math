\section{Решение алгебраических уравнений третей степени}
Любое кубическое уравнение общего вида
$ax^{3}+bx^{2}+cx+d=0$
при помощи замены переменной
$x=y-{\frac{b}{3a}}$
может быть приведено к указанной выше канонической форме с коэффициентами

$p={\frac  {c}{a}}-{\frac  {b^{2}}{3a^{2}}}={\frac  {3ac-b^{2}}{3a^{2}}}$,

$q={\frac  {2b^{3}}{27a^{3}}}-{\frac  {bc}{3a^{2}}}+{\frac  {d}{a}}={\frac  {2b^{3}-9abc+27a^{2}d}{27a^{3}}}.$

$Q=\left({\frac  {p}{3}}\right)^{3}+\left({\frac  {q}{2}}\right)^{2}.$

Если все коэффициенты кубического уравнения вещественны, то и Q вещественно, и по его знаку можно определить тип корней:

Q > 0 — один вещественный корень и два сопряжённых комплексных корня.

Q = 0 — один однократный вещественный корень и один двукратный, или, если p = q = 0, то один трёхкратный вещественный корень.

Q < 0 — три вещественных корня.

По формуле Кардано, корни кубического уравнения в канонической форме равны:

$\displaystyle y_{1}=\alpha +\beta ,$

$y_{{2,3}}=-{\frac  {\alpha +\beta }{2}}\pm i{\frac  {\alpha -\beta }{2}}{\sqrt  {3}},$ где

$\alpha =\sqrt[ {3}]{-{\frac  {q}{2}}+{\sqrt  {Q}}}$,

$\beta =\sqrt[ {3}]{-{\frac  {q}{2}}-{\sqrt  {Q}}}$,

Дискриминант многочлена $y^{3}+py+q$ при этом равен $\Delta =-108Q$.

Решим данное уравнение:

\begin{sagesilent}
	x, a, b, c, d = var("x", "a", "b", "c", "d")
	y = a*x**3 + b*x**2 + c*x + d;
	y = x**3 -5*x**2 + 7*x - 3 
	
	a = 1
	b = -5
	c = 7
	d = -3
	
	var("p", "q")
	p = (3*a*c - b**2)/(3*a**2)
	q = ((2*b**3)/(27*a**3)) - ((b*c)/(3*a**2))+ (d/a)
	Q = ((p/3)**3) + ((q/2)**2)
\end{sagesilent}

$\sage{y}=0$

Вычислим коэффициенты:

$p=\sage{p}$

$q=\sage{q}$

$Q=\sage{Q}$

\begin{sagesilent}
	alpha = ((-q/2) + sqrt(Q))**(1/3)
	beta = ((-q/2) - sqrt(Q))**(1/3)
	
	epsilon = -1/2 + (sqrt(-3))/2
	
	z = [alpha + beta
	, alpha*epsilon + beta*epsilon**2
	, beta*epsilon + alpha*epsilon**2
	]
	
	def to_x(_z, _a):
	    return _z - _a/3
	    
	def trigonometric_representation(z):
        phi = atan2(z.imag(), z.real()).n(digits=4)
        z_abs = z.abs().n(digits=4)
        z_trig = z_abs*(cos(phi, hold=True) + I*sin(phi, hold=True))
	
	def exponential_representation(z):
        phi = atan2(z.imag(), z.real()).n(digits=4)
        z_abs = z.abs().n(digits=4)
	
\end{sagesilent}
~\\
Q = 0 — один однократный вещественный корень и один двукратный.
~\\
Корни уравнения в алгебраическом, тригонометрическом и экспоненциальном представлениях:

$x_0=\sage{to_x(z[0], b/a)}$

$x_0=\sage{3} *cos(0) + 3*i*sin(0)$

$x_0=\sage{to_x(z[0], b/a)}* e^{0*i}$

$x_{1,2}=\sage{to_x(z[1], b/a).real()}$

$x_{1,2}=\sage{to_x(z[1], b/a).real()} * cos(0) + \sage{to_x(z[1], b/a).real()} * i * sin(0)$

$x_{1,2}=\sage{to_x(z[1], b/a).real()} * e^{0*i}$

\begin{sagesilent}
    f(x) = a*x**3 + b*x**2 + c*x + d
    grapfic = plot(f(x), -2, 4)
    grapfic += point((find_root(f, 1, 1.5), 0), size = 50, color = 'red')
    grapfic += point((find_root(f, 2.9, 3.1), 0), size = 50, color = 'red')
\end{sagesilent}

\sageplot{plot(grapfic)}