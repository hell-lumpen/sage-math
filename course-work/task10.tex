\section{Численные методы — Метод касательных}
\begin{sagesilent}
	 def newton(f, x0):
	    df = f.derivative()
	    x1 = x0 - f(x = x0) / df(x = x0)
	    x2 = x1 - f(x = x1) / df(x = x1)
	    l = df(x = x1)*(x - x1) + f(x = x1)
	    while abs(x2.n(digits=10) - x1.n(digits=10)) > 0.0000000001:
	        x1 = x2 - f(x = x2) / df(x = x2)
	        x2 = x1
	    return x2, l
\end{sagesilent}

\begin{sagesilent}
	x = var("x")
	func = (x**2 + 3*cos(1/2*x**2) - 1) / (3*x) - 1/3
	root1, l = newton(func, 2)
	root2, l1 = newton(func, 2.5)
	plot1 = plot(func, 1, 3)
	plot2 = plot(func, 1, 3)
	plot2 += plot(l, 1, 3, color="red")
	plot2 += plot(l1, 1, 3, color="red")
	plot2 += point((root1, 0), size = 30)
	plot2 += point((root2, 0), size = 30)
\end{sagesilent}

Метод касательных (метод Ньютона) предназначен для приближенного нахождения нулей функции.

Чтобы численно решить уравнение $f(x)=0$ методом простой итерации, его необходимо привести к эквивалентному уравнению: $x=\varphi(x)$, где $\varphi$  — сжимающее отображение.

Для наилучшей сходимости метода в точке очередного приближения $x^*$ должно выполняться условие $\varphi '(x^*)=0$. Решение данного уравнения ищут в виде $\varphi (x)=x+\alpha (x)f(x)$, тогда:

$\varphi '(x^{*})=1+\alpha '(x^{*})f(x^{*})+\alpha (x^{*})f'(x^{*})=0.$

В предположении, что точка приближения «достаточно близка» к корню $\tilde  {x}$ и что заданная функция непрерывна $(f(x^{*})\approx f({\tilde {x}})=0)$, окончательная формула для $\alpha(x)$ такова:

$\alpha (x)=-{\frac {1}{f'(x)}}$.

С учётом этого функция $\varphi (x)$ определяется:

$\varphi (x)=x-{\frac {f(x)}{f'(x)}}.$

При некоторых условиях эта функция в окрестности корня осуществляет сжимающее отображение.

Дана функция $f(x) = \sage{func}$. Вычислить с помощью метода Ньютона корни соответствующего уравнения.

\sageplot[width=10cm]{plot(plot1)}

Запустим два раза метод Ньютона для двух окрестностей. Первая точка будет $x = 2$, а вторая $x = 2.5$. Приближенно получили $x_1 = \sage{root1.n(digits=6)}$, $x_2 = \sage{root2.n(digits=6)}$

\sageplot{plot2}
