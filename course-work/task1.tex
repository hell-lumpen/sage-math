\section{Исследование графиков}

Дана функция $f(x) = sin(2*x^3)^2/x^3 + x$. 

\sageplot{plot(sin(2*x**3)**2/x**3 + x, -2, 2)}

\subsection{Область определения функции}

Область определения функции --- $x \in \mathbb{R}$, $x \neq 0$ т.к. функция содержит в знаменателе $x^3$. \\$f(x):D(f(x)) = (-\infty; +\infty) /\ 0$

\subsection{Является ли функция четной или нечетной, является ли периодической}

Функция $f:X \to \mathbb{R}$ называется чётной, если справедливо равенство:\\
$f(-x)=f(x),\quad \forall x\in X$\\
Функция называется нечётной, если справедливо равенство:\\
$f(-x)=-f(x),\quad \forall x\in X$\\
функция называется периодической с периодом  $T\neq 0$, если для каждой точки $x$ из её области определения точки $x+T$ и $x-T$ также принадлежат её области определения, и для них выполняется равенство $f(x)=f(x+T)=f(x-T)$.

Данная функция является нечетной и не является периодической.

\subsection{Точки пересечения графика с осями координат}

По области определения функции $x \neq 0$ значит функция никогда не пересекает ось ординат.
Найдем точки пересечения графика функции $f(x)$ с осью абсцисс. Для этого решим уравнение: $sin(2*x^3)^2/x^3 + x = 0$. Функция никогда не пересекает ось абсцисс.

\subsection{Промежутки знакопостоянства}

График функции никогда не пересекает ось абсцисс, но есть точка разрыва $x = 0$. При $x > 0$ значение функции в любой точке положительное, а при $x < 0$ --- отрицательное.

\subsection{Промежутки возрастания и убывания}



\subsection{Точки экстремума и значения в этих точках}

График производной функции:

\begin{sagesilent}
	var("x")
	y(x) = sin(2*x**3)**2/x**3 + x
	d_f = y.diff(x)
	graf = plot(d_f, -2, 2)
	graf += point((find_root(d_f, -1, -0.5), 0), size = 50, color = 'red')
	graf += point((find_root(d_f, 0.5, 1), 0), size = 50, color = 'red')
\end{sagesilent}

\sageplot{plot(graf)}

\subsection{Непрерывность. Наличие точек разрыва и их классификация}

Если в точке имеются конечные пределы, но они не равны $f(x_0+0) \neq f(x_0-0)$, то $x_0$ называется точкой разрыва первого рода.

Точками разрыва второго рода называются точки, в которых хотя бы один из односторонних пределов равен $\infty$ или не существует.

Данная функция имеет точку разрыва первого рода при $x=0$

\subsection{Асимптоты. Найти необходимые пределы, построить асимптоты на графике}

Вертикальные асимптоты проходят через точки разрыва (когда предел функции в этой точке стремится к бесконечности). 

Горизонтальные асимптоты проходят через точку с координатой $х$, которая является пределом функции при устремлении аргумента функции к бесконечности.

Наклонные асимптоты задаются графиком прямой $kx+b$, где за наклон будет отвечать предел отношения функции к аргументу, где последний стремится к бесконечности. За смещение отвечает предел функции $f(x)-kx$, где $х$ стремится к бесконечности.

\begin{sagesilent}
y = sin(2*x**3)**2/x**3 + x
asimpt = plot(y, -3, 3)
if (y.limit(x = 0) == oo):
    asimpt += plot(0, linestyle="dashed", color="#FF0000")


if (y.limit(x = oo) != oo):
    asimpt += plot(y.limit(x = oo), linestyle="dashed", color="#00FF00")


f1 = y(x)/x
k = f1.limit(x = oo)
f2 = y(x)-k*x
b = f2.limit(x = oo)
if (k != oo and b != oo):
    asimpt += plot(k*x+b, xmin = -2, xmax = 2, linestyle="dashed", color="#9ACD32")

\end{sagesilent}

\sageplot{plot(asimpt)}

