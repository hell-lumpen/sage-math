\section{СЛАУ}
\begin{equation*}
	\begin{cases}
		2x_1 - 3x_2 + x_3 = 2, 
		\\
		x_1 + 5x_2 - 4x_3 = -5,
		\\
		4x_1 + x_2 - 3x_3 = -4, 
	\end{cases}
\end{equation*}

\subsection{Метод Крамера}

\begin{sagesilent}
	matrix_a = matrix([[2,-3,1],[1,5,-4],[4,1,-3]])
	matrix_b = matrix([[2],[-5],[-4]])
\end{sagesilent}

Решим данную систему методом Крамера.
Получаем матрицу коэффициентов
$A = \sage{matrix_a}$ и столбец свободных членов $B = \sage{matrix_b}$.
$D = det(A) = \begin{vmatrix}
	2 & -3 & 1\\
	1 & 5 & -4\\
	4 & 1 & -3\\
\end{vmatrix} = \sage{matrix_a.det()}$.
$D \neq 0$, следовательно система уравнений имеет решение. В определителях столбец коэффициентов при соответствующей неизвестной заменяется столбцом свободных членов системы.

$D_1 = \begin{vmatrix}
	2 & -3 & 1\\
	-5 & 5 & -4\\
	-4 & 1 & -3\\
\end{vmatrix} = -10$, 
$D_2 = \begin{vmatrix}
	2 & 2 & 1\\
    1 & -5 & -4\\
    4 & -4 & -3\\
\end{vmatrix} = -12$,
$D_3 = \begin{vmatrix}
	2 & -3 & 2\\
    1 & 5 & -5\\
    4 & 1 & -4\\
\end{vmatrix} = -20$.

Найдем $x_1 = D_1/D = 5$, $x_2 = D_2/D = 6$, $x_3 = D_3/D = 10$.
\subsection{Метод Гаусса}

Сначала необходимо проверить совместность системы по теореме Кронекера-Капелли. Для того чтобы линейная система являлась совместной, необходимо и достаточно, чтобы ранг расширенной матрицы этой системы был равен рангу её основной матрицы. 
Система уравнений разрешима тогда и только тогда, когда $ \operatorname {rank} A=\operatorname {rank} (A|B)$, где $(A|B)$ — расширенная матрица, полученная из матрицы $A$ приписыванием столбца $B$.

\begin{sagesilent}
	matrix_a = matrix([[2,-3,1],[1,5,-4],[4,1,-3]])
	matrix_b = matrix([[2],[-5],[-4]])
	
	new_mat = block_matrix([matrix_a, matrix_b], nrows=1)

	a_rank = matrix_a.rank()
	new_rank = new_mat.rank()
	new_mat = new_mat.echelon_form()
\end{sagesilent}
Ранг матрицы коэффициентов $ \operatorname {rank} A= \sage{matrix_a.rank()}$.

Ранг расширенной матрицы $ \operatorname {rank} (A|B) = \sage{new_mat.rank()}$.

Ранги матриц равны следовательно система совместна.
\begin{sagesilent}
	line = 3
	column  = line + 1 
	
	x_3 = new_mat[line-1][column-1]/new_mat[line-1][column-2]
	x_2 = (new_mat[line-2][column-1] - x_3*new_mat[line-2][column-2])/new_mat[line-2][column-3]
	x_1 = (new_mat[line-3][column-1]- x_3*new_mat[line-3][column-2] - x_2*new_mat[line-3][column-3])/new_mat[line-3][column-4]
	
	ans = matrix([[x_1],[x_2],[x_3]])
\end{sagesilent}

Далее необходимо привести матрицу к диагональному виду 
$\sage{block_matrix([matrix_a, matrix_b], nrows=1)} \sim \begin{pmatrix}
	4 & 1 & -3& \vline & -4\\
	0 & -7/2 & 5/2 & \vline & 4\\
	0 & 0 & 1/7 & \vline & 10/7\\
\end{pmatrix} \sim \sage{ans}$.
