\section{Численные методы — Метод касательных}
\begin{sagesilent}
def newton(f, a, b):
    x1 = a
    x2 = (a + b) / 2
    df = f.derivative()
    while abs(x2 - x1) >= 0.0000001:
        x1 = x2
        x2 = x1 - f(x=x1)/df(x=x1)
    return x1
\end{sagesilent}

\begin{sagesilent}
func = (x**2 + 3*cos(1/2*x**2) - 1) / (3*x) - 1/3

x_1 = newton(func, 1.5, 2.0)
x_2 = newton(func, 2.8, 2.9)

p1 = plot(func, xmin = 1, xmax = 3)
p1 += point((x_1, 0), color="red", size=30, zorder=20)
p1 += point((x_2, 0), color="red", size=30, zorder=20)

x, y = var("x,y")
x1 = 1.5
x2 = (1.5 + 2.0)/2
grafic1 = plot(p1)
df = func.derivative()
df2 = df.diff(x)

tangent1 = []
while abs(x2 - x1) >= 0.0000001:
    x1 = x2
    x2 = x1 - func(x=x1) / df(x=x1)
    l = df(x1)*(x - x1) + func(x=x1)
    i = 2
    if i:
        tangent1.append(plot(l, xmin = 1, xmax = 3, color="red") + text((r"$y={}$" + "\n" + r"$x_1={}, x_2={}$").format(latex(l), x1, x2), (2, -1), fontsize=12, color="black"))
        i -= 1
   
x1 = 2.8
x2 = (2.8 + 2.9)/2
df = func.derivative()

tangent2 = []
while abs(x2 - x1) >= 0.0000001:
    x1 = x2
    x2 = x1 - func(x=x1) / df(x=x1)
    l = df(x1)*(x - x1) + func(x=x1)
    i = 2
    if i:
        tangent2.append(plot(l, xmin = 1, xmax = 3, color="red") + text((r"$y={}$" + "\n" + r"$x_1={}, x_2={}$").format(latex(l), x1, x2), (2, -1), fontsize=12, color="black"))
        i -= 1
\end{sagesilent}

Метод касательных (метод Ньютона) предназначен для приближенного нахождения нулей функции.

Чтобы численно решить уравнение $f(x)=0$ методом простой итерации, его необходимо привести к эквивалентному уравнению: $x=\varphi(x)$, где $\varphi$  — сжимающее отображение.

Для наилучшей сходимости метода в точке очередного приближения $x^*$ должно выполняться условие $\varphi '(x^*)=0$. Решение данного уравнения ищут в виде $\varphi (x)=x+\alpha (x)f(x)$, тогда:

$\varphi '(x^{*})=1+\alpha '(x^{*})f(x^{*})+\alpha (x^{*})f'(x^{*})=0.$

В предположении, что точка приближения «достаточно близка» к корню $\tilde  {x}$ и что заданная функция непрерывна $(f(x^{*})\approx f({\tilde {x}})=0)$, окончательная формула для $\alpha(x)$ такова:

$\alpha (x)=-{\frac {1}{f'(x)}}$.

С учётом этого функция $\varphi (x)$ определяется:

$\varphi (x)=x-{\frac {f(x)}{f'(x)}}.$

При некоторых условиях эта функция в окрестности корня осуществляет сжимающее отображение.

Дана функция $f(x) = \sage{func}$. Вычислить с помощью метода Ньютона корни соответствующего уравнения.

\sageplot[width=10cm]{plot(func, 1, 3)}

Запустим два раза метод Ньютона для двух окрестностей. Первая точка будет $x = 2$, а вторая $x = 2.5$. Приближенно получили $x_1 = \sage{x_1.n(digits=6)}$, $x_2 = \sage{x_2.n(digits=6)}$
\newpage
\textbf{Несколько итераций алгоритма для каждого корня:}
~\\
\sageplot[width=8.5cm]{grafic1 + tangent1[0]}
\sageplot[width=8.5cm]{grafic1 + tangent2[0]}
~\\
\sageplot[width=8.5cm]{grafic1 + tangent1[1]}
\sageplot[width=8.5cm]{grafic1 + tangent2[1]}
~\\
Проверим условия сходимости метода Ньютона с помощью теоремы Кантаровича.

$\frac{1}{f'(x)} < A$ (условие того, что $f'(x)$ существует и не равна 0;)

$\frac{f(x)}{f'(x)} < B$ (условие того, что $f(x)$ ограничена;)

$\exists f''(x)$ и $|f''(x)| \leqslant C \leqslant \frac{1}{2AB}$;

Проверим все вышеупомянутые условия:
\begin{center}
	$A = \frac{1}{f'(2)} = \sage{abs(1/df(x = 2)).n(digits = 4)}$
	
	$B = \frac{f(2)}{f'(2)} = \sage{abs(func(x = 2)/df(x = 2)).n(digits = 4)}$
	
	$C = |f''(x)| = \sage{abs(df2(x = 2)).n(digits = 4)}$
	
	$C \leqslant \frac{1}{2AB}$
	
	$\sage{abs(df2(x = 2)).n(digits = 4)} \leqslant \sage{(1 / (2 * abs(1/df(x = 2)) * abs(func(x = 2)/df(x = 2)))).n(digits = 4)}$ --- значит метод Ньютона применим.
\end{center}