\section{НОД двух полиномов}

Для полиномов $f(x)$ и $g(x)$ по алгоритму Евклида найти наибольший общий делитель (НОД).
Представить НОД в виде: $d(x) = f(x)u(x) + g(x)v(x)$

Алгоритм Евклида (алгоритм последовательного деления) нахождения
наибольшего общего делителя многочленов $a$ и $b:$

$r_{k}$ — это остаток от деления предпредыдущего числа на предыдущее, а предпоследнее делится на последнее нацело, то есть:

$a=bq_{0}+r_{1},$

$b=r_{1}q_{1}+r_{2},$

$r_{1}=r_{2}q_{2}+r_{3},$

$\cdots$

$r_{k-2}=r_{k-1}q_{k-1}+r_{k},$

$\cdots$

$r_{n-2}=r_{n-1}q_{n-1}+r_{n},$

$r_{n-1}=r_{n}q_{n}.$

Таким образом последний ненулевой остаток является наибольшим общим
делителем исходных многочленов $a$ и $b$.

\begin{sagesilent}
R.<x> = QQ[]
f = x**4 + x**3 - 2*x**2 - x**1 - 1
g = x**3 + x**2 - x - 1

def GCD(a, b):
    if a == 0:
        return (b, 0, 1)
    else:
        div, x, y = GCD(b % a, a)
        
    return(div, y - (b // a) * x, x)
    
    
div, u, v = GCD(g, f)

if(f != 0 and g != 0):
    _gcd = div.monic() # Return this polynomial divided by its leading coefficient
    norm_coeff = _gcd / div
    v = v * norm_coeff
    u = u * norm_coeff
else:
    _gcd = div
    u = 0 
    v = 0
\end{sagesilent}
~\\
Даны полиномы $f = \sage{f}$ и $g = \sage{g}$.

$gcd(f, g) = \sage{_gcd} $
~\\
~\\
По теореме Безу НОД можно представить в виде: $gcd(f, g) = f*u + v*g$:

$gcd(f, g) = (\sage{f}) * (\sage{u}) + (\sage{v}) * (\sage{g}) = \sage{_gcd}$