\section{Линейные преобразования и характеристическое уравнение матрицы}

\begin{sagesilent}
	A = Matrix([[3, -1, -2], [2, 1, -3], [1, 0, -1]])
	
	S = Matrix([[1, 1, 1], [1, 1, 0], [0, 2, 1]])
	e1, e2, e3 = var("e_1 e_2 e_3")
	ve = vector([e1, e2, e3])
	vec = []
	for i, row in enumerate(S * ve):
	    vec.append(row)
	    
	S = S.transpose()
	A_new = S^(-1)*A*S
	
	R.<x> = QQ[]
	
	phi_A = det(A - x*identity_matrix(3)).monic()
	
	lam = var("lambda_")
	f = phi_A(lam)
	
	sols = solve(f, lam, solution_dict=True)

\end{sagesilent}
Дано преобразование в базис: $A = \sage{A}$, 
$\begin{cases}
	e_0' = \sage{vec[0]}, 
	\\
    e_1' = \sage{vec[1]}, 
	\\
    e_2' = \sage{vec[2]}, 
\end{cases}$

Пусть преобразование задаётся матрицей A. А матрица перехода в другой базис задаётся матрицей S.
Тогда искомая матрица $A'$ находится по формуле:

$A' = S^{-1}AS$

Тогда матрица $A' = \sage{A_new}$.

Найдем характеристический полином. Он задается формулой:

 $\phi_A(\lambda) =|A - \lambda E|$
 
Характеристический полином матриц $A$ и $A'$ совпадает и равен: 

$\phi_A = \sage{f}$

Найдем собственные числа $\phi_A(\lambda) = 0$

$\sage{f} = 0$

Отсюда собственные числа равны:

$\lambda_0 = \sage{sols[0][lam].n(digits=3)}$

$\lambda_1 = \sage{sols[1][lam].n(digits=3)}$

$\lambda_2 = \sage{sols[2][lam].n(digits=3)}$