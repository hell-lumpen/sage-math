% размер шрифта: 14pt
% и указываем размер страницы - у нас это A4 (a4paper)
% подключаем макеты оформления extreport или extarticle, 
% чтобы кегль выше 12pt отображался адекватно
\documentclass[14pt, a4paper]{extarticle}

%поля: 
%слева: 2.5
%справа: 1.5
%сверху: 2.5
%снизу: 2.5
\usepackage[left=2.5cm, right=1.5cm, top=2.5cm, bottom=2.5cm]{geometry}

% установка междустрочного интервала
% закомментировано, т.к. нас устраивает пока настройки по умолчанию
% \linespread{1.3}

% кодировка файла
% не должно объявляться более одного раза
% кодировку подключаем в первую очередь
% иначе могут быть весьма неприятные непонятные ошибки
\usepackage[utf8x]{inputenc}
% шрифты
\usepackage[T2A]{fontenc}
% языки
\usepackage[english,russian]{babel}


% пакеты
\usepackage{amsmath} % align
\usepackage{amssymb}
\usepackage[usenames]{color}

% мета-данные. Т.е. данные, описывающие данные.
% в текущем случае не выводятся никуда (не будут видны в pdf)
\title{Курсовой проект по дисциплине Математический практикум}
\author{Выполнил: Автор, группа}

% подключение пакетов для ссылок и для SageTeX
\usepackage{hyperref}
\usepackage{sagetex}

% размер отступа для первой строки абзаца
\setlength{\sagetexindent}{10ex}

% определяем короткий псевдоним для объявления секции с автоматической нумерацией
\renewcommand{\thesection}{\number\numexpr\value{section}-1\relax}
% для subsection, например, задаём формат записи заголовка "номер секции.номер подсекции-1"
\renewcommand{\thesubsection}{\thesection.\number\numexpr\value{subsection}-1\relax}
\renewcommand{\thesubsubsection}{\thesubsection.\number\numexpr\value{subsubsection}-1\relax}

% установка начальных значений счетчиков секций и пр.
\setcounter{secnumdepth}{1}
% \setcounter{chapter}{1}  % этого счетчика нет в шаблоне 'article'
\setcounter{section}{1}


% открываем тег документа
\begin{document}
	
	\include{"TitlePage"}
	\tableofcontents
	
	\section{Исследование графиков}

Дана функция $f(x) = sin(2*x^3)^2/x^3 + x$. 

\sageplot{plot(sin(2*x**3)**2/x**3 + x, -2, 2)}

\subsection{Область определения функции}

Область определения функции --- $x \in \mathbb{R}$, $x \neq 0$ т.к. функция содержит в знаменателе $x^3$. \\$f(x):D(f(x)) = (-\infty; +\infty) /\ 0$

\subsection{Является ли функция четной или нечетной, является ли периодической}

Функция $f:X \to \mathbb{R}$ называется чётной, если справедливо равенство:\\
$f(-x)=f(x),\quad \forall x\in X$\\
Функция называется нечётной, если справедливо равенство:\\
$f(-x)=-f(x),\quad \forall x\in X$\\
функция называется периодической с периодом  $T\neq 0$, если для каждой точки $x$ из её области определения точки $x+T$ и $x-T$ также принадлежат её области определения, и для них выполняется равенство $f(x)=f(x+T)=f(x-T)$.

Данная функция является нечетной и не является периодической.

\subsection{Точки пересечения графика с осями координат}

По области определения функции $x \neq 0$ значит функция никогда не пересекает ось ординат.
Найдем точки пересечения графика функции $f(x)$ с осью абсцисс. Для этого решим уравнение: $sin(2*x^3)^2/x^3 + x = 0$. Функция никогда не пересекает ось абсцисс.

\subsection{Промежутки знакопостоянства}

График функции никогда не пересекает ось абсцисс, но есть точка разрыва $x = 0$. При $x > 0$ значение функции в любой точке положительное, а при $x < 0$ --- отрицательное.

\begin{sagesilent}
	intervals_of_constancy = plot(0, xmin=-0.5, xmax=0.5, ymin = 0, ymax = 0, axes = False)
	intervals_of_constancy += circle((0, 0), 0.008, rgbcolor="black")
	y_margin = 0.1
	intervals_of_constancy  += text("-", (-0.25 , y_margin), color="black", fontsize=25) 
	intervals_of_constancy  += text("+", (0.25 , y_margin), color="black", fontsize=25)
	x = var("x")
	y = sin(2*x**3)**2/x**3 + x
	func = y.diff(x)
	plot_interval = plot(y, 0.5, 1.5)
	x1 = find_root(func, 0.8, 1.0)
	x2 = find_root(func, 1.1, 1.2)
	x3 = find_root(func, 1.2, 1.4)
\end{sagesilent}

\sageplot{plot(intervals_of_constancy)}

\subsection{Промежутки возрастания и убывания}

\sageplot{plot(plot_interval)}

Исследуем функцию на промежутке $[0.5; 1.5]$.
Точки перегиба:

$x_0 = \sage{x1}$

$x_1 = \sage{x2}$

$x_2 = \sage{x3}$

Функция возрастает на промежутках $(0.5; \sage{x1})$ и

$(\sage{x2};\sage{x3})$.

Убывает при $x \in (\sage{x1}; \sage{x2})$.
\newpage
\subsection{Точки экстремума и значения в этих точках}

График производной функции:

\begin{sagesilent}
	var("x")
	y(x) = sin(2*x**3)**2/x**3 + x
	d_f = y.diff(x)
	graf = plot(d_f, 0.5, 1.5)
	graf += point((x1, 0), size = 30, color = 'black')
	graf += point((x2, 0), size = 30, color = 'black')
	graf += point((x3, 0), size = 30, color = 'black')
\end{sagesilent}

\sageplot{plot(graf)}

\textbf{Необходимые условия существования экстремума:}

Пусть точка $x_{0}$ является точкой экстремума функции $f$, определенной в некоторой окрестности точки $x_{0}$.
Тогда либо производная $f'(x_{0})$ не существует, либо $f'(x_{0})=0$.

Эти условия не являются достаточными, так, функция может иметь нуль производной в точке, но эта точка может не быть точкой экстремума, а являться, скажем, точкой перегиба.

\textbf{Достаточные условия существования экстремума:}

1. Если функция непрерывна в окрестности точки или  не существует и производная  при переходе через точку  меняет свой знак, тогда в точке  функция  имеет экстремум, причем это минимум, если при переходе через точку  производная меняет свой знак с минуса на плюс; максимум, если при переходе через точку  производная меняет свой знак с плюса на минус.

2. Если функция непрерывна в окрестности точки, $f'(x_{0}) = 0$, а $f''(x_{0}) \neq 0$, то в $x_0$  достигается экстремум, причем, если $f''(x_{0}) > 0$ , то в точке функция имеет минимум; если $f''(x_{0}) < 0$, то в точке  функция  достигает максимум.

\begin{sagesilent}
	var("x")
	y(x) = sin(2*x**3)**2/x**3 + x
	d_f = y.diff(x)
	d_f2 = d_f.diff(x)
\end{sagesilent}
~\\

Исследуем $x_0 = \sage{x1}$. $f''(x_0) = \sage{d_f2(x1).n(digits = 5)}$. Значение меньше нуля --- точка является локальным максимумом.

Далее $x_1 = \sage{x2}$. $f''(x_1) = \sage{d_f2(x2).n(digits = 5)}$. Значение больше нуля --- точка является локальным минимумом.

Наконец, исследуем $x_2 = \sage{x3}$. $f''(x_2) = \sage{d_f2(x3).n(digits = 5)}$. Значение меньше нуля --- точка является локальным максимумом.

\subsection{Непрерывность. Наличие точек разрыва и их классификация}

Если в точке имеются конечные пределы, но они не равны $f(x_0+0) \neq f(x_0-0)$, то $x_0$ называется точкой разрыва первого рода.

Точками разрыва второго рода называются точки, в которых хотя бы один из односторонних пределов равен $\infty$ или не существует.

Данная функция имеет точку разрыва первого рода при $x=0$

\subsection{Асимптоты. Найти необходимые пределы, построить асимптоты на графике}

Вертикальные асимптоты проходят через точки разрыва (когда предел функции в этой точке стремится к бесконечности). 

Горизонтальные асимптоты проходят через точку с координатой $х$, которая является пределом функции при устремлении аргумента функции к бесконечности.

Наклонные асимптоты задаются графиком прямой $kx+b$, где за наклон будет отвечать предел отношения функции к аргументу, где последний стремится к бесконечности. За смещение отвечает предел функции $f(x)-kx$, где $х$ стремится к бесконечности.

\begin{sagesilent}
y = sin(2*x**3)**2/x**3 + x
asimpt = plot(y, -3, 3)
if (y.limit(x = 0) == oo):
    asimpt += plot(0, linestyle="dashed", color="#FF0000")


if (y.limit(x = oo) != oo):
    asimpt += plot(y.limit(x = oo), linestyle="dashed", color="#00FF00")


f1 = y(x)/x
k = f1.limit(x = oo)
f2 = y(x)-k*x
b = f2.limit(x = oo)
if (k != oo and b != oo):
    asimpt += plot(k*x+b, xmin = -2, xmax = 2, linestyle="dashed", color="#9ACD32")

\end{sagesilent}

\sageplot{plot(asimpt)}


	\include{task2}
	\section{Матрицы — Матричные уравнения}

\begin{sagesilent}
    A = matrix(QQ, [[-2, 4, -6], [-1, 0, 2], [4, 4, 2]])
    B = matrix(QQ, [[2, 8, -3], [-2, 2, 4], [1, 0, 2]])
\end{sagesilent}

Дано уравнение:

$1/2 * X * \sage{A}^{T} - 3*E = 1/4 * \sage{B}^{2}$
~\\

Решение уравнения сводится к двум вариантам умножения:
~\\

\begin{minipage}{0.4\textwidth}
	$A*X = B$
	
	$A^{-1}* A*X = A^{-1}*B$
	
	$X = A^{-1}*B$
\end{minipage}
\hfill
\begin{minipage}{0.4\textwidth}
	$X*A=B$
	
	$X*A*A^{-1} = B*A^{-1}$
	
	$X = B*A^{-1}$
	
\end{minipage}

\begin{sagesilent}
	A = matrix(QQ, [[-2, 4, -6], [-1, 0, 2], [4, 4, 2]])
	B = matrix(QQ, [[2, 8, -3], [-2, 2, 4], [1, 0, 2]])
	E = matrix(QQ, [[3, 0, 0], [0, 3, 0], [0, 0, 3]])
	A = A.transpose() / 2
	B = B**2 / 4 + E
	X = B * A.inverse()
\end{sagesilent}

~\\

$X * \sage{A} = \sage{B}$

~\\

$X = \sage{X}$

	\section{Решение алгебраических уравнений третей степени}
Любое кубическое уравнение общего вида
$ax^{3}+bx^{2}+cx+d=0$
при помощи замены переменной
$x=y-{\frac{b}{3a}}$
может быть приведено к указанной выше канонической форме с коэффициентами

$p={\frac  {c}{a}}-{\frac  {b^{2}}{3a^{2}}}={\frac  {3ac-b^{2}}{3a^{2}}}$,

$q={\frac  {2b^{3}}{27a^{3}}}-{\frac  {bc}{3a^{2}}}+{\frac  {d}{a}}={\frac  {2b^{3}-9abc+27a^{2}d}{27a^{3}}}.$

$Q=\left({\frac  {p}{3}}\right)^{3}+\left({\frac  {q}{2}}\right)^{2}.$

Если все коэффициенты кубического уравнения вещественны, то и Q вещественно, и по его знаку можно определить тип корней:

Q > 0 — один вещественный корень и два сопряжённых комплексных корня.

Q = 0 — один однократный вещественный корень и один двукратный, или, если p = q = 0, то один трёхкратный вещественный корень.

Q < 0 — три вещественных корня.

По формуле Кардано, корни кубического уравнения в канонической форме равны:

$\displaystyle y_{1}=\alpha +\beta ,$

$y_{{2,3}}=-{\frac  {\alpha +\beta }{2}}\pm i{\frac  {\alpha -\beta }{2}}{\sqrt  {3}},$ где

$\alpha =\sqrt[ {3}]{-{\frac  {q}{2}}+{\sqrt  {Q}}}$,

$\beta =\sqrt[ {3}]{-{\frac  {q}{2}}-{\sqrt  {Q}}}$,

Дискриминант многочлена $y^{3}+py+q$ при этом равен $\Delta =-108Q$.

Решим данное уравнение:

\begin{sagesilent}
	x, a, b, c, d = var("x", "a", "b", "c", "d")
	y = a*x**3 + b*x**2 + c*x + d;
	y = x**3 -5*x**2 + 7*x - 3 
	
	a = 1
	b = -5
	c = 7
	d = -3
	
	var("p", "q")
	p = (3*a*c - b**2)/(3*a**2)
	q = ((2*b**3)/(27*a**3)) - ((b*c)/(3*a**2))+ (d/a)
	Q = ((p/3)**3) + ((q/2)**2)
\end{sagesilent}

$\sage{y}=0$

Вычислим коэффициенты:

$p=\sage{p}$

$q=\sage{q}$

$Q=\sage{Q}$

\begin{sagesilent}
	alpha = ((-q/2) + sqrt(Q))**(1/3)
	beta = ((-q/2) - sqrt(Q))**(1/3)
	
	epsilon = -1/2 + (sqrt(-3))/2
	
	z = [alpha + beta
	, alpha*epsilon + beta*epsilon**2
	, beta*epsilon + alpha*epsilon**2
	]
	
	def to_x(_z, _a):
	    return _z - _a/3
	    
	def trigonometric_representation(z):
        phi = atan2(z.imag(), z.real()).n(digits=4)
        z_abs = z.abs().n(digits=4)
        z_trig = z_abs*(cos(phi, hold=True) + I*sin(phi, hold=True))
	
	def exponential_representation(z):
        phi = atan2(z.imag(), z.real()).n(digits=4)
        z_abs = z.abs().n(digits=4)
	
\end{sagesilent}
~\\
Q = 0 — один однократный вещественный корень и один двукратный.
~\\
Корни уравнения в алгебраическом, тригонометрическом и экспоненциальном представлениях:

$x_0=\sage{to_x(z[0], b/a)}$

$x_0=\sage{3} *cos(0) + 3*i*sin(0)$

$x_0=\sage{to_x(z[0], b/a)}* e^{0*i}$

$x_{1,2}=\sage{to_x(z[1], b/a).real()}$

$x_{1,2}=\sage{to_x(z[1], b/a).real()} * cos(0) + \sage{to_x(z[1], b/a).real()} * i * sin(0)$

$x_{1,2}=\sage{to_x(z[1], b/a).real()} * e^{0*i}$

\begin{sagesilent}
    f(x) = a*x**3 + b*x**2 + c*x + d
    grapfic = plot(f(x), -2, 4)
    grapfic += point((find_root(f, 1, 1.5), 0), size = 50, color = 'red')
    grapfic += point((find_root(f, 2.9, 3.1), 0), size = 50, color = 'red')
\end{sagesilent}

\sageplot{plot(grapfic)}
	\section{Решение алгебраических уравнений четвертой степени}
\begin{sagesilent}
	x = var("x")
	f_x = x**4 + 3*x**3 - 4*x**2 - 3*x + 3
	a = 3
	b = -4
	c = -3
	d = 3
	y = var("y")
	polynom = f_x(x = y - 3/4).expand().simplify_full()
	
	pqr = {'p':-59/8, 'q': 51/8, 'r': 525/256}
	var("s p q r")
	resolvent = 2*s**3 - p*s**2 - 2*r*s + r*p - q**2/4

	poly_s_n = resolvent(**pqr)
	
	sols = solve(poly_s_n, s)  
	s_0 = sols[2].rhs()
	
	var("y s p q")
	poly_y_1 = y**2 - y*sqrt(2*s - p) + q/(2*sqrt(2*s - p)) + s
	poly_y_2 = y**2 + y*sqrt(2*s - p) - q/(2*sqrt(2*s - p)) + s
	
	poly_y_1_n = poly_y_1(**pqr, s=s_0)
	poly_y_2_n = poly_y_2(**pqr, s=s_0)
	
	sols = solve(poly_y_1_n, y)
	sols.extend(solve(poly_y_2_n, y))
	
	sols = solve(f_x, x)
	graf = plot(f_x, -4, 2)
	for i, sol in enumerate(sols):
	    a = sol.rhs()
	    graf += point((a, 0), size = 40, color = 'black')
	
\end{sagesilent}
Дано уравнение: $\sage{f_x} = 0$

Метод Феррари состоит из двух этапов.

На первом этапе уравнения вида 
\begin{center}
	$a_0x^4 + a_1x^3 + a_2x^2 + a_3x + a_4 = 0$,
    
    где $a_0, a_1, a_2, a_3, a_4$ - произвольные числа, причем $a_0 \neq 0$.
	\end{center} 

приводятся к уравнениям четвертой степени, у которых отсутствует член с третьей степенью неизвестного.

На втором этапе полученные уравнения решаются при помощи разложения на множители, однако для того, чтобы найти требуемое разложение на множители, приходится решать кубические уравнения, например методом Кардано.

Пусть уравнение имеет вид $x^4+ax^3 + bx^2 + cx + d = 0$:

Произведем замену:
$y = x -\frac{a}{4}$, где $a$ --- коэффициент при переменной в третьей степени уравнения с коэффициентом при старшей степени равным 1.

$ y = x - 3/4 $, тогда заданное уравнение имеет вид: 

$\sage{polynom} = 0$.

Найдем значение коэффициентов $p, q, r$ для исходного уравнения, чтобы уравнение приняло вид: $y^4 + py^3 + qy + r = 0$:

$p = b- \frac{3a^2}{8} = \sage{b - (3*a**2)/8}$;

$q = \frac{a^3}{8} - \frac{ab}{2} + c = \sage{(a**3) / 8 - (a*b)/2 + c}$;

$r = \frac{-3a^4}{256} + \frac{a^2b}{16} - \frac{ca}{4} + d = \sage{(-3*a**4 )/ 256 +( a**2*b) / 16 - (c*a) / 4 + d}$.

Добавив и вычитая в левой части уравнения выражение $2sy^2 + s^2$ получим:
$\sage{resolvent}$

Это выражение называется кубическая резольвента. Найдем хотя бы одно значение $s$ из уравнения $\sage{poly_s_n} = 0$.

$s_0 = \sage{s_0}$

Получаем два уравнения из которого вычисляется $y$, а затем с помощью обратной замены --- $x$.

$\sage{poly_y_1_n} = 0$ ; $\sage{poly_y_2_n} = 0$
~\\

$x_0 = \sage{sols[0].rhs().n(digits = 4)}$

$x_1 = \sage{sols[1].rhs().n(digits = 4)}$

$x_2 = \sage{sols[2].rhs().n(digits = 4)}$

$x_3 = \sage{sols[3].rhs().n(digits = 4)}$

\sageplot{plot(graf)}
	\section{НОД двух полиномов}

Для полиномов $f(x)$ и $g(x)$ по алгоритму Евклида найти наибольший общий делитель (НОД).
Представить НОД в виде: $d(x) = f(x)u(x) + g(x)v(x)$

Алгоритм Евклида (алгоритм последовательного деления) нахождения
наибольшего общего делителя многочленов $a$ и $b:$

$r_{k}$ — это остаток от деления предпредыдущего числа на предыдущее, а предпоследнее делится на последнее нацело, то есть:

$a=bq_{0}+r_{1},$

$b=r_{1}q_{1}+r_{2},$

$r_{1}=r_{2}q_{2}+r_{3},$

$\cdots$

$r_{k-2}=r_{k-1}q_{k-1}+r_{k},$

$\cdots$

$r_{n-2}=r_{n-1}q_{n-1}+r_{n},$

$r_{n-1}=r_{n}q_{n}.$

Таким образом последний ненулевой остаток является наибольшим общим
делителем исходных многочленов $a$ и $b$.

\begin{sagesilent}
R.<x> = QQ[]
f = x**4 + x**3 - 2*x**2 - x**1 - 1
g = x**3 + x**2 - x - 1

def GCD(a, b):
    if a == 0:
        return (b, 0, 1)
    else:
        div, x, y = GCD(b % a, a)
        
    return(div, y - (b // a) * x, x)
    
    
div, u, v = GCD(g, f)

if(f != 0 and g != 0):
    _gcd = div.monic() # Return this polynomial divided by its leading coefficient
    norm_coeff = _gcd / div
    v = v * norm_coeff
    u = u * norm_coeff
else:
    _gcd = div
    u = 0 
    v = 0
\end{sagesilent}
~\\
Даны полиномы $f = \sage{f}$ и $g = \sage{g}$.

$gcd(f, g) = \sage{_gcd} $
~\\
~\\
По теореме Безу НОД можно представить в виде: $gcd(f, g) = f*u + v*g$:

$gcd(f, g) = (\sage{f}) * (\sage{u}) + (\sage{v}) * (\sage{g}) = \sage{_gcd}$
	\section{Линейные преобразования и характеристическое уравнение матрицы}

\begin{sagesilent}
	A = Matrix([[3, -1, -2], [2, 1, -3], [1, 0, -1]])
	
	S = Matrix([[1, 1, 1], [1, 1, 0], [0, 2, 1]])
	e1, e2, e3 = var("e_1 e_2 e_3")
	ve = vector([e1, e2, e3])
	vec = []
	for i, row in enumerate(S * ve):
	    vec.append(row)
	    
	S = S.transpose()
	A_new = S^(-1)*A*S
	
	R.<x> = QQ[]
	
	phi_A = det(A - x*identity_matrix(3)).monic()
	
	lam = var("lambda_")
	f = phi_A(lam)
	
	sols = solve(f, lam, solution_dict=True)

\end{sagesilent}
Дано преобразование в базис: $A = \sage{A}$, 
$\begin{cases}
	e_0' = \sage{vec[0]}, 
	\\
    e_1' = \sage{vec[1]}, 
	\\
    e_2' = \sage{vec[2]}, 
\end{cases}$

Пусть преобразование задаётся матрицей A. А матрица перехода в другой базис задаётся матрицей S.
Тогда искомая матрица $A'$ находится по формуле:

$A' = S^{-1}AS$

Тогда матрица $A' = \sage{A_new}$.

Найдем характеристический полином. Он задается формулой:

 $\phi_A(\lambda) =|A - \lambda E|$
 
Характеристический полином матриц $A$ и $A'$ совпадает и равен: 

$\phi_A = \sage{f}$

Найдем собственные числа $\phi_A(\lambda) = 0$

$\sage{f} = 0$

Отсюда собственные числа равны:

$\lambda_0 = \sage{sols[0][lam].n(digits=3)}$

$\lambda_1 = \sage{sols[1][lam].n(digits=3)}$

$\lambda_2 = \sage{sols[2][lam].n(digits=3)}$
	\section{Упрощение уравнений фигур 2-го порядка в пространстве}

\begin{sagesilent}
	coefs = [0, 0, 3, -6, 4, 0, -1, -1, 1, -10]
	a11, a22, a33, a12, a13, a23, a14, a24, a34, a44 = coefs
	a14 /= 2 
	a24 /= 2
	a34 /= 2
	a12 /= 2
	a13 /= 2
	a23 /= 2
	
	var("x y z")
	f_src = a11*x**2 + a22*y**2 + a33*z**2\
	+ 2*a12*x*y + 2*a13*x*z + 2*a23*y*z\
	+ 2*a14*x + 2*a24*y + 2*a34*z\
	+ a44
	
	
	var('x y z')
	
	plot_ = implicit_plot3d(f_src, (-10,10), (-10, 10), (-10, 10), plot_points=100)
	
	A = Matrix([
	[a11, a12, a13]
	, [a12, a22, a23]
	, [a13, a23, a33]
	])
	
	a = Matrix([[a14], [a24], [a34]])
	
	phi_A = A.charpoly()
	lam = var("lambda_")
	f = phi_A(lam)
	
	ssols = A.eigenvalues()
	resvec = A.eigenvectors_left()
	res = []
	res_vec = []
	for i, ssol in enumerate(ssols):
	    res.append(ssol.n(digits=4))
	    res_vec.append(resvec[i][1])
	    
	vectors = []
	for i in range (3):
	    vectors.append(res_vec[i][0].n(digits=4) / (sqrt((res_vec[i][0][0].n(digits=4))**2 + (res_vec[i][0][1].n(digits=4))**2 + (res_vec[i][0][2].n(digits=4))**2)))
	    
	    
	S = Matrix([vectors[0], vectors[1], vectors[2]])
	
	a_ = S.transpose()*a

	new_f = res[0] * x**2 + res[1] * y**2 + res[2] * z**2 + 2*a_[0][0]*x + 2*a_[1][0] * y + 2* a_[2][0] * z + a44
	
	new1_f = res[0] * (x + a_[0][0] / res[0])**2 + res[1] * (y + a_[1][0] / res[1])**2 + res[2] * (z + a_[2][0] / res[2])**2
	new1_f2 = (a_[0][0])**2 / res[0] + (a_[1][0])**2 / res[1] + (a_[2][0])**2 / res[2] - a44
	
	new1_f = new1_f / new1_f2
	var('X Y Z')
	kanon_f = new1_f(x = X - a_[0][0] / res[0], y = Y - a_[1][0] / res[1], z = Z - a_[2][0] / res[2])	
	
	new_plot = implicit_plot3d(kanon_f - 1, (-10,10), (-10, 10), (-10, 10), plot_points=100)
	
	    
\end{sagesilent}

Дана функции $u(x, y, z) = \sage{f_src}$
\begin{center}
	\sageplot[clip, width=14cm]{plot(plot_)}
\end{center}

Составим матрицу $A$ этой квадратичной формы и столбец коэффициентов
линейной формы: 

\begin{center}
	$A = \sage{A}$, $B = \sage{B}$
\end{center}

Найдем собственные числа матрицы $A$. Характеристический полином:
\begin{center}
	$\phi_A = \sage{f}$
\end{center}

Корни характеристического полинома --- собственные числа матрицы $A$:
\begin{center}
	$\lambda_0 = \sage{res[0]}, \lambda_1 = \sage{res[1]}, \lambda_2 = \sage{res[2]}$
\end{center}

Найдем взаимно перпендикулярные единичные собственные векторы $s_1, s_2, s_3$, соответствующие собственным числам, и составим из них матрицу $S = (s_1|s_2|s_3).$ Так как все собственные числа простые, поэтому для каждого корня найдем ненулевое решение однородной СЛАУ $(A− \lambda_iE)l_i = 0$, для каждого $i = 0, 1, 2$.
\begin{center}
	$v_0 = \sage{res_vec[0][0].n(digits=4)}$;
	
	$v_1 = \sage{res_vec[1][0].n(digits=4)}$;
	
	$v_2 = \sage{res_vec[2][0].n(digits=4)}$. 
\end{center}

Нормируем векторы. Чтобы нормировать вектор, разделим координаты вектора на его
модуль.
\begin{center}
	$v_0 = \sage{vectors[0]}$;
	
	$v_1 = \sage{vectors[1]}$;
	
	$v_2 = \sage{vectors[2]}$. 
\end{center}

Составим матрицу из векторов для перехода к новому базису:
\begin{center}
    $S = \sage{S}$
\end{center}

Вычислить столбец коэффициентов линейной формы $a'=S^Ta$
\begin{center}
	$a' = \sage{a_}$
\end{center}

Составим уравнение <<почти>> канонического вида:
\begin{center}
	$f(x, y, z) = \lambda_0(x)^2+\lambda_1(y)^2+\lambda_2(z)^2+ 2a'_0 x + 2a'_1 y+ 2a'_2 z+a.$
	~\\
	~\\
	$u(x, y, z) = \sage{new_f}.$
\end{center}

Дополним до полного квадрата линейные и квадратичные члены, перенесем вправо свободные члены и разделим все уравнение на правую часть, чтобы справа оставалась $1$:
\begin{center}
	$\lambda_0(x + \frac{a'_0}{\lambda_0})^2 + \lambda_1(y + \frac{a'_1}{\lambda_1})^2 + \lambda_2(z + \frac{a'_2}{\lambda_2})^2 = \frac{(a'_0)^2}{\lambda_0} + \frac{(a'_1)^2}{\lambda_1} + \frac{(a'_2)^2}{\lambda_2} - a$
	~\\
	~\\
	$\sage{new1_f} = \sage{new1_f2}$
	~\\ 
	~\\
	$\sage{new1_f} = 1$
\end{center}

Теперь заменим значения в скобках на соответствующие переменные $X, Y, Z$. Получим канонический вид заданного уравнения фигуры второго порядка:

\begin{center}

	$\sage{kanon_f} = 1$

	Фигура, построенная по каноническому уравнению:

	\sageplot[clip, width=14cm]{plot(new_plot)}
\end{center}



	\section{Численные методы — Интегралы}

\begin{sagesilent}
	import numpy
	var("x")
	
	y = (8*x - arctan(2*x)) / (4*x**2 + 1)
	a = 0 # нижний предел
	b = e # верхний предел
	
	plot1 = plot(y, a, b)
	
	fill_plot1 = plot(y, a, b, fill = True, fillcolor = "red", title = "Необходимо найти площадь закрашенной фигуры")
	
	# метод прямоугольников
	
	dx = 0.1 # приращение x
	
	graph_rectangle = plot(y, a, b)
	result_rectangle = 0
	rectangles = []
	iteration_rect = []
	
	for i, xi in enumerate(numpy.arange(a, b , dx)):
	    dy = y(xi) # приращение y
	    result_rectangle += dx*dy; # площадь маленького прямоугольника из которых будет состоять общая площадь
	
	    text_ = text(r"$i={}, curr={}, result={}$".format(i, str((dx*dy).n(digits=4)),str(result_rectangle.n(digits=4))), (1,-0.5), fontsize=12, color="black")
	
	    rectangles.append(
	        polygon2d([(xi, 0),
	        (xi + dx, 0),
	        (xi + dx, dy),
	        (xi, dy)])
	        )
	    graph_rectangle += plot(rectangles[-1])

	    if (i < 5):
	        iteration_rect.append(graph_rectangle)
	    
	
# метод трапеций

dx_ = 0.3

graph_trapezoid = plot(y, a, b)
result_trapezoid = 0
trapezoids = []
iteration_trap = []


for i, xi in enumerate(numpy.arange(a, b , dx_)):
    dy1 = y(xi)
    dy2 = y(xi + dx_)
    result_trapezoid += (dy1 + dy2) / 2 * dx_ # площадь маленькой трапеции из которых будет состоять общая площадь

    trapezoids.append(
        polygon2d([(xi, 0),
        (xi + dx_, 0),
        (xi + dx_, dy2),
        (xi, dy1)])
    )

    graph_trapezoid += plot(trapezoids[-1])

    if i < 5:
        iteration_trap.append(graph_trapezoid)

result_sage, e = numerical_integral(y, a, b)

\end{sagesilent}

\begin{minipage}{0.4\textwidth}
	Найти площадь закрашенной фигуры методом трапеций и прямоугольников. Площадь численно равна интегралу:
	$\int\limits_{\sage{a}}^{\sage{b}} \sage{y} dx$.
\end{minipage}
\hfill
\begin{minipage}{0.4\textwidth}
	\sageplot[width=7cm]{plot(fill_plot1)}
\end{minipage}

\subsection{Метод прямоугольников}

Отрезок интегрирования разбивается на равные части отрезки длины: $\Delta x = \frac{b−a}{n}$. Данная величина называется шагом разбиения. С помощью шага разбиения можно регулировать точность вычисления интеграла. В результате получим точки:
$x_0 = a < x_1 < x_2 < \dots < x_{n−1} = b$.
Тогда значение интеграла будет суммой площадей всех маленьких прямоугольников:

\begin{center}
	$\sum\limits_{i=1}^n  f(x_n)\cdot \Delta x_n$

\sageplot[width=7cm]{plot(iteration_rect[0])}
\sageplot[width=7cm]{plot(iteration_rect[1])}

\sageplot[width=7cm]{plot(iteration_rect[2])}
\sageplot[width=7cm]{plot(iteration_rect[3])}
\end{center}

Значение интеграла полученное методом прямоугольников с шагом разбиения $\sage{dx.n(digits = 2)}$.

\begin{center}
	$\int\limits_{\sage{a}}^{\sage{b}} \sage{y} dx = \sage{result_rectangle}$.
\end{center}

\subsection{Метод трапеций}

Отрезок интегрирования разбивается на равные части отрезки длины: $\Delta x = \frac{b−a}{n}$.

Численно интеграл будет равен сумме площадей маленьких прямоугольных трапеций:

\begin{center}
	$\sum\limits_{i=1}^n  \frac{f(x_n) + f(x_n + \Delta x_n)}{2} \cdot \Delta x_n$
	
	\sageplot[width=7cm]{plot(iteration_trap[0])}
	\sageplot[width=7cm]{plot(iteration_trap[1])}
	
	\sageplot[width=7cm]{plot(iteration_trap[2])}
	\sageplot[width=7cm]{plot(iteration_trap[3])}
\end{center}

Значение интеграла полученное методом трапеций с шагом разбиения $\sage{dx_.n(digits = 2)}$.

\begin{center}
	$\int\limits_{\sage{a}}^{\sage{b}} \sage{y} dx = \sage{result_trapezoid}$.
\end{center}

Результат, полученный встроенными средствами Sage:
\begin{center}
	$\int\limits_{\sage{a}}^{\sage{b}} \sage{y} dx = \sage{result_sage}$.
\end{center}
	\section{Численные методы — Метод касательных}
\begin{sagesilent}
	 def newton(f, x0):
	    df = f.derivative()
	    x1 = x0 - f(x = x0) / df(x = x0)
	    x2 = x1 - f(x = x1) / df(x = x1)
	    l = df(x = x1)*(x - x1) + f(x = x1)
	    while abs(x2.n(digits=10) - x1.n(digits=10)) > 0.0000000001:
	        x1 = x2 - f(x = x2) / df(x = x2)
	        x2 = x1
	    return x2, l
\end{sagesilent}

\begin{sagesilent}
	x = var("x")
	func = (x**2 + 3*cos(1/2*x**2) - 1) / (3*x) - 1/3
	root1, l = newton(func, 2)
	root2, l1 = newton(func, 2.5)
	plot1 = plot(func, 1, 3)
	plot2 = plot(func, 1, 3)
	plot2 += plot(l, 1, 3, color="red")
	plot2 += plot(l1, 1, 3, color="red")
	plot2 += point((root1, 0), size = 30)
	plot2 += point((root2, 0), size = 30)
\end{sagesilent}

Метод касательных (метод Ньютона) предназначен для приближенного нахождения нулей функции.

Чтобы численно решить уравнение $f(x)=0$ методом простой итерации, его необходимо привести к эквивалентному уравнению: $x=\varphi(x)$, где $\varphi$  — сжимающее отображение.

Для наилучшей сходимости метода в точке очередного приближения $x^*$ должно выполняться условие $\varphi '(x^*)=0$. Решение данного уравнения ищут в виде $\varphi (x)=x+\alpha (x)f(x)$, тогда:

$\varphi '(x^{*})=1+\alpha '(x^{*})f(x^{*})+\alpha (x^{*})f'(x^{*})=0.$

В предположении, что точка приближения «достаточно близка» к корню $\tilde  {x}$ и что заданная функция непрерывна $(f(x^{*})\approx f({\tilde {x}})=0)$, окончательная формула для $\alpha(x)$ такова:

$\alpha (x)=-{\frac {1}{f'(x)}}$.

С учётом этого функция $\varphi (x)$ определяется:

$\varphi (x)=x-{\frac {f(x)}{f'(x)}}.$

При некоторых условиях эта функция в окрестности корня осуществляет сжимающее отображение.

Дана функция $f(x) = \sage{func}$. Вычислить с помощью метода Ньютона корни соответствующего уравнения.

\sageplot[width=10cm]{plot(plot1)}

Запустим два раза метод Ньютона для двух окрестностей. Первая точка будет $x = 2$, а вторая $x = 2.5$. Приближенно получили $x_1 = \sage{root1.n(digits=6)}$, $x_2 = \sage{root2.n(digits=6)}$

\sageplot{plot2}

	\section{Линейный оператор и базисы}

\begin{sagesilent}
def to_x(_z, _a):
    return _z - _a/3
\end{sagesilent}

\begin{sagesilent}
import copy as cp
A = Matrix([[2, 0, -1], [1, 1, -1], [-1, 0, 2]])
B = Matrix([[0], [0], [0]])
R.<x> = QQ[]
phi_A = det(A - x*identity_matrix(3)).monic()
# метод Кардано для поиска собственных чисел
x, a, b, c, d = var("x", "a", "b", "c", "d")
a = 1
b = -5
c = 7
d = -3
y = a*x**3 + b*x**2 + c*x + d;
var("p", "q")
p = (3*a*c - b**2)/(3*a**2)
q = ((2*b**3)/(27*a**3)) - ((b*c)/(3*a**2))+ (d/a)
Q = ((p/3)**3) + ((q/2)**2)
alpha = ((-q/2) + sqrt(Q))**(1/3)
beta = ((-q/2) - sqrt(Q))**(1/3)
epsilon = -1/2 + (sqrt(-3))/2

res = [alpha + beta
, alpha*epsilon + beta*epsilon**2
, beta*epsilon + alpha*epsilon**2
]

result_phi = []
result_matrix = []
vectors = []
result_vectors = []

for i, z_i in enumerate(res):
    x_i = to_x(_z = z_i, _a = b/a)
    result_phi.append(x_i.n(digits=2).real())


for i, lam1 in enumerate(result_phi):
    mat = A - (identity_matrix(3) * lam1)
    mat = mat.echelon_form()
    result_matrix.append(mat)

for i, lam1 in enumerate(result_phi):
    vectors.append(result_matrix[i].right_kernel().basis())
    result_vectors.append(vectors[i][0])

new_basis = Matrix([[1, 1, 0], [1, -1, 1], [0, 1, -1]])
new_basis_result = []
for i, lam1 in enumerate(result_vectors):
    new_basis_result.append(new_basis.inverse() * result_vectors[i])
    
S = Matrix([[1, 1, -1], [1, 0, 1], [0, 1, 0]])
S = S.transpose()
v2 = Matrix(QQ, [0.00, 1.0, 0.00])
A = Matrix([[2,0,-1],[1,1,-1],[-1,0,2]])
J = S.inverse() * A * S
J_sage = A.jordan_form()

M = (J * A.inverse()).inverse()

coefficients_ = phi_A.coefficients()
for i in range(4):
    coefficients_[i] = coefficients_[i] * -1
    
A1 = (coefficients_[3]*A**2 + coefficients_[2]*A + coefficients_[1]*identity_matrix(3))/-coefficients_[0]
A3 = coefficients_[2]*A*A + coefficients_[1]*A + coefficients_[0]


\end{sagesilent}

Линейный оператор A задан в каноническом базисе матрицей.
\begin{equation*}
	A = \sage{A}
\end{equation*}

Найдем собственные числа матрицы, решив методом Кардано характеристическое уравнение:

\begin{center}
	$\phi_A = \sage{phi_A};$
	
    $x_0 = \sage{result_phi[0]};$
    
	$x_1 = \sage{result_phi[1]};$
	
	$x_2 = \sage{result_phi[2]}.$
\end{center}

Приведем получившиеся матрицы к ступенчатому виду, чтобы найти собственные векторы $A$.

\begin{center}
	$A_{\lambda_0} = \sage{result_matrix[0]};$
	отсюда собственный вектор $\sage{result_vectors[0]}$
	
	$A_{\lambda_1} = \sage{result_matrix[1]};$
	отсюда собственный вектор $\sage{result_vectors[1]}$
	
	$A_{\lambda_2} = \sage{result_matrix[2]};$
	отсюда собственный вектор $\sage{v2}$
\end{center}

Найдем найденные собственные векторы в базисе $B = \sage{new_basis}$. Для этого воспользуемся формулой:

\begin{center}
	$V' = B^{-1}  V$
	
	$V'_{\lambda_0} = \sage{new_basis_result[0]}$
	
	$V'_{\lambda_1} = \sage{new_basis_result[1]}$

	$V'_{\lambda_2} = \sage{(new_basis.inverse() * v2.transpose()).transpose()}$
\end{center}

Жордановой матрицей называют блочно-диагональную матрицу, на диагонали которой стоят жордановы клетки.

Составим матрицу из собственных векторов.
\begin{center}
	$S = \sage{S}$
\end{center}

Для перехода к жордановой форме транспонируем $S$ и воспользуемся формулой:
\begin{center}
	$J = S^TAS$
\end{center}

Отсюда жорданова форма матрицы $A$:
\begin{center}
 	$J = \sage{J}$
\end{center} 

Найдем матрицу перехода от матрицы $A$ к жорданову базису 
~\\
воспользовавшись формулой:

\begin{center}
	$M = (BA^{-1})^{-1}$
\end{center}

Исходя из этого матрица перехода к жордановой форме:

\begin{center}
	$M_J = (JA^{-1})^{-1} = \sage{M}$
\end{center}

Вычислим $A^{-1}$ и $A^3$ по теореме Кэли - Гамильтона.

Теорема Кэли - Гамильтона утверждает, что любая квадратная матрица удовлетворяет своему характеристическому уравнению, и как следствие,
~\\
 обуславливает существование аннулирующего многочлена. Выпишем
~\\ 
характеристическое 
уравнение:
\begin{center}
	$\phi_A = \sage{phi_A}$, откуда следует, что
	
	$A^3 = 5A^2 - 7A + 3E$
\end{center}

Вычисляя $A^2$ как $A*A$ и суммируя с остальными слагаемыми, получаем:
\begin{center}
	$A^3 = \sage{5*A*A} - \sage{7*A} + \sage{3*identity_matrix(3)}$;
	
	$A^3 = \sage{5*A*A - 7*A + 3*identity_matrix(3)}$
\end{center}
~\\
~\\

Чтобы вычислить $A^{-1}$ домножим слева все уравнение на  $A^{-1}$:

\begin{center}
		$A^3 - 5A^2 + 7A - 3E = 0 $ $\vert$ $\cdot A^{-1}$
		~\\
		$A^2 - 5A + 7E - 3A^{-1} = 0$
		~\\
		$A^{-1} = (A^2 - 5A + 7E) / 3$
		~\\
		~\\
		$A^{-1} = (\sage{A*A} - \sage{5*A} + \sage{7*identity_matrix(3)}) / 3$
		~\\
		~\\
		$A^{-1} = \sage{(A*A - 5*A + 7*identity_matrix(3)) / 3}$
\end{center}


	\include{"Sources"}
	
	\setcounter{page}{2} % начать нумерацию с номера 2
	
\end{document}
