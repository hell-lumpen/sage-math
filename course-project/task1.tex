\section{Исследование графиков}

Дана функция $f(x) = sin(2*x^3)^2/x^3 + x$. 

\sageplot{plot(sin(2*x**3)**2/x**3 + x, -2, 2)}

\subsection{Область определения функции}

Область определения функции --- $x \in \mathbb{R}$, $x \neq 0$ т.к. функция содержит в знаменателе $x^3$. \\$f(x):D(f(x)) = (-\infty; +\infty) /\ 0$

\subsection{Является ли функция четной или нечетной, является ли периодической}

Функция $f:X \to \mathbb{R}$ называется чётной, если справедливо равенство:\\
$f(-x)=f(x),\quad \forall x\in X$\\
Функция называется нечётной, если справедливо равенство:\\
$f(-x)=-f(x),\quad \forall x\in X$\\
функция называется периодической с периодом  $T\neq 0$, если для каждой точки $x$ из её области определения точки $x+T$ и $x-T$ также принадлежат её области определения, и для них выполняется равенство $f(x)=f(x+T)=f(x-T)$.

Данная функция является нечетной и не является периодической.

\subsection{Точки пересечения графика с осями координат}

По области определения функции $x \neq 0$ значит функция никогда не пересекает ось ординат.
Найдем точки пересечения графика функции $f(x)$ с осью абсцисс. Для этого решим уравнение: $sin(2*x^3)^2/x^3 + x = 0$. Функция никогда не пересекает ось абсцисс.

\subsection{Промежутки знакопостоянства}

График функции никогда не пересекает ось абсцисс, но есть точка разрыва $x = 0$. При $x > 0$ значение функции в любой точке положительное, а при $x < 0$ --- отрицательное.

\begin{sagesilent}
	intervals_of_constancy = plot(0, xmin=-0.5, xmax=0.5, ymin = 0, ymax = 0, axes = False)
	intervals_of_constancy += circle((0, 0), 0.008, rgbcolor="black")
	y_margin = 0.1
	intervals_of_constancy  += text("-", (-0.25 , y_margin), color="black", fontsize=25) 
	intervals_of_constancy  += text("+", (0.25 , y_margin), color="black", fontsize=25)
	x = var("x")
	y = sin(2*x**3)**2/x**3 + x
	func = y.diff(x)
	plot_interval = plot(y, 0.5, 1.5)
	x1 = find_root(func, 0.8, 1.0)
	x2 = find_root(func, 1.1, 1.2)
	x3 = find_root(func, 1.2, 1.4)
\end{sagesilent}

\sageplot{plot(intervals_of_constancy)}

\subsection{Промежутки возрастания и убывания}

\sageplot{plot(plot_interval)}

Исследуем функцию на промежутке $[0.5; 1.5]$.
Точки перегиба:

$x_0 = \sage{x1}$

$x_1 = \sage{x2}$

$x_2 = \sage{x3}$

Функция возрастает на промежутках $(0.5; \sage{x1})$ и

$(\sage{x2};\sage{x3})$.

Убывает при $x \in (\sage{x1}; \sage{x2})$.
\newpage
\subsection{Точки экстремума и значения в этих точках}

График производной функции:

\begin{sagesilent}
	var("x")
	y(x) = sin(2*x**3)**2/x**3 + x
	d_f = y.diff(x)
	graf = plot(d_f, 0.5, 1.5)
	graf += point((x1, 0), size = 30, color = 'black')
	graf += point((x2, 0), size = 30, color = 'black')
	graf += point((x3, 0), size = 30, color = 'black')
\end{sagesilent}

\sageplot{plot(graf)}

\textbf{Необходимые условия существования экстремума:}

Пусть точка $x_{0}$ является точкой экстремума функции $f$, определенной в некоторой окрестности точки $x_{0}$.
Тогда либо производная $f'(x_{0})$ не существует, либо $f'(x_{0})=0$.

Эти условия не являются достаточными, так, функция может иметь нуль производной в точке, но эта точка может не быть точкой экстремума, а являться, скажем, точкой перегиба.

\textbf{Достаточные условия существования экстремума:}

1. Если функция непрерывна в окрестности точки или  не существует и производная  при переходе через точку  меняет свой знак, тогда в точке  функция  имеет экстремум, причем это минимум, если при переходе через точку  производная меняет свой знак с минуса на плюс; максимум, если при переходе через точку  производная меняет свой знак с плюса на минус.

2. Если функция непрерывна в окрестности точки, $f'(x_{0}) = 0$, а $f''(x_{0}) \neq 0$, то в $x_0$  достигается экстремум, причем, если $f''(x_{0}) > 0$ , то в точке функция имеет минимум; если $f''(x_{0}) < 0$, то в точке  функция  достигает максимум.

\begin{sagesilent}
	var("x")
	y(x) = sin(2*x**3)**2/x**3 + x
	d_f = y.diff(x)
	d_f2 = d_f.diff(x)
\end{sagesilent}
~\\

Исследуем $x_0 = \sage{x1}$. $f''(x_0) = \sage{d_f2(x1).n(digits = 5)}$. Значение меньше нуля --- точка является локальным максимумом.

Далее $x_1 = \sage{x2}$. $f''(x_1) = \sage{d_f2(x2).n(digits = 5)}$. Значение больше нуля --- точка является локальным минимумом.

Наконец, исследуем $x_2 = \sage{x3}$. $f''(x_2) = \sage{d_f2(x3).n(digits = 5)}$. Значение меньше нуля --- точка является локальным максимумом.

\subsection{Непрерывность. Наличие точек разрыва и их классификация}

Если в точке имеются конечные пределы, но они не равны $f(x_0+0) \neq f(x_0-0)$, то $x_0$ называется точкой разрыва первого рода.

Точками разрыва второго рода называются точки, в которых хотя бы один из односторонних пределов равен $\infty$ или не существует.

Данная функция имеет точку разрыва первого рода при $x=0$

\subsection{Асимптоты. Найти необходимые пределы, построить асимптоты на графике}

Вертикальные асимптоты проходят через точки разрыва (когда предел функции в этой точке стремится к бесконечности). 

Горизонтальные асимптоты проходят через точку с координатой $х$, которая является пределом функции при устремлении аргумента функции к бесконечности.

Наклонные асимптоты задаются графиком прямой $kx+b$, где за наклон будет отвечать предел отношения функции к аргументу, где последний стремится к бесконечности. За смещение отвечает предел функции $f(x)-kx$, где $х$ стремится к бесконечности.

\begin{sagesilent}
y = sin(2*x**3)**2/x**3 + x
asimpt = plot(y, -3, 3)
if (y.limit(x = 0) == oo):
    asimpt += plot(0, linestyle="dashed", color="#FF0000")


if (y.limit(x = oo) != oo):
    asimpt += plot(y.limit(x = oo), linestyle="dashed", color="#00FF00")


f1 = y(x)/x
k = f1.limit(x = oo)
f2 = y(x)-k*x
b = f2.limit(x = oo)
if (k != oo and b != oo):
    asimpt += plot(k*x+b, xmin = -2, xmax = 2, linestyle="dashed", color="#9ACD32")

\end{sagesilent}

\sageplot{plot(asimpt)}

